\documentclass[12pt,a4paper]{article}
\usepackage[utf8]{inputenc}
\usepackage[T1]{fontenc}
\usepackage[brazilian]{babel}
\usepackage{indentfirst}
\usepackage{graphicx}
\usepackage{float}
\usepackage{geometry}
\usepackage{titlesec}
\usepackage{hyperref}

\geometry{
    a4paper,
    left=3cm,
    right=2cm,
    top=3cm,
    bottom=2cm
}

\titleformat{\section}{\normalfont\Large\bfseries}{\thesection}{1em}{}
\titleformat{\subsection}{\normalfont\large\bfseries}{\thesubsection}{1em}{}

\begin{document}

\begin{titlepage}
    \begin{center}
        \vspace*{3cm}
        {\Large\textbf{Split}}\\[0.5cm]
        {\large\textbf{Planejamento Estratégico}}\\[7cm]
        
        João Henrique\\
        Marcelle Andrade\\
        Nicolas Martins \\ 
        Pedro Balsamão\\
        Vinicius Corrêa de Assis\\[4cm]
        
        \today
    \end{center}
\end{titlepage}

\tableofcontents
\newpage

\section{Sobre a Empresa}

\subsection{Descrição do Produto e Funcionalidades}
O Split nasceu de uma realidade que todos já vivenciamos: o momento constrangedor da divisão de contas em grupos. Quem nunca passou pela situação desconfortável onde algumas pessoas dividiram uma porção, enquanto outras compartilharam diferentes itens, gerando uma complexa matemática que frequentemente resulta em discussões, constrangimentos e até mesmo atritos entre amigos?

Imagine um grupo de dez amigos em um restaurante. Três pessoas dividiram uma porção de batatas fritas, mas apenas duas delas, junto com uma quarta pessoa, compartilharam uma cerveja. Enquanto isso, outros pediram pratos individuais e bebidas diferentes. No final da noite, o que deveria ser um momento de descontração se transforma em uma sessão tensa de cálculos e discussões sobre valores. Este cenário, tão comum em nossa sociedade, muitas vezes resulta em pessoas pagando mais do que deveriam, outras se sentindo constrangidas, e algumas até evitando futuros encontros para não passar pela mesma situação.

O Split transforma completamente esta experiência. Nossa solução inovadora utiliza tecnologia de ponta para eliminar todo o desconforto social associado à divisão de contas. Através de um sistema intuitivo e automatizado, cada pessoa paga exatamente pelo que consumiu, sem discussões, sem calculadoras e, mais importante, sem prejudicar as amizades.

\subsubsection{Como Funciona}
Nosso aplicativo utiliza inteligência artificial para escanear a comanda e automaticamente identificar todos os itens consumidos. Durante a refeição, os usuários podem facilmente marcar quem consumiu cada item, seja individual ou compartilhado, através de uma interface amigável e intuitiva. O sistema então calcula automaticamente o valor devido por cada pessoa, considerando inclusive divisões complexas de itens compartilhados e taxas de serviço.

\subsubsection{Principais Funcionalidades}
\begin{itemize}
    \item Escaneamento inteligente de comandas com IA
    \item Sistema de marcação individual de itens consumidos
    \item Divisão proporcional automática de itens compartilhados
    \item Interface intuitiva de seleção de itens por pessoa
    \item Grupos personalizados para divisões específicas
    \item Cálculo automático de gorjetas e taxas de serviço
    \item Histórico detalhado de consumo individual
    \item Sistema de pagamento integrado
    \item Compartilhamento fácil do resumo da divisão
\end{itemize}

\subsection{Membros e Responsabilidades}
A equipe é composta por profissionais especializados em diferentes áreas:

\begin{itemize}
    \item João Henrique - Desenvolvimento Backend
    \item Marcelle Andrade - UX/UI Design
    \item Nicolas Martins - Desenvolvimento Backend
    \item Pedro Balsamão - Desenvolvimento Frontend
    \item Vinicius Corrêa de Assis - Gerência de Produto
\end{itemize}

\subsection{Missão}
Transformar momentos de confraternização em experiências verdadeiramente agradáveis, eliminando o desconforto da divisão de contas e fortalecendo relacionamentos através de uma solução tecnológica simples e eficaz.

\subsection{Visão}
Ser reconhecida como a solução definitiva para divisão de contas no Brasil, tornando-se sinônimo de harmonia e praticidade em momentos de confraternização, presente em todos os estabelecimentos do país.

\subsection{Valores}
\begin{itemize}
    \item Transparência nas relações
    \item Justiça na divisão
    \item Preservação de amizades
    \item Simplicidade de uso
    \item Inovação tecnológica
    \item Confiabilidade
\end{itemize}

\section{Análise SWOT}

\subsection{Forças (Strengths)}
\begin{itemize}
    \item Produto nacional adaptado ao mercado brasileiro
    \item Interface intuitiva e amigável
    \item Automatização de processos manuais
    \item Integração com sistemas existentes
    \item Processamento em tempo real
\end{itemize}

\subsection{Fraquezas (Weaknesses)}
\begin{itemize}
    \item Dependência de conexão com internet
    \item Necessidade de treinamento inicial
    \item Limitações do MVP inicial
    \item Base de usuários inicial reduzida
\end{itemize}

\subsection{Oportunidades (Opportunities)}
\begin{itemize}
    \item Mercado em expansão de pagamentos digitais
    \item Demanda crescente por soluções automatizadas
    \item Possibilidade de parcerias estratégicas
    \item Expansão para outros segmentos
\end{itemize}

\subsection{Ameaças (Threats)}
\begin{itemize}
    \item Concorrentes estabelecidos no mercado
    \item Mudanças regulatórias no setor financeiro
    \item Resistência à adoção de novas tecnologias
    \item Instabilidade econômica
\end{itemize}

\section{Análise de Concorrência}

\subsection{Identificação dos Principais Concorrentes}

\subsubsection{Concorrente 1 - Splitwise}
\begin{itemize}
    \item \textbf{Funcionalidades}: 
        \begin{itemize}
            \item Divisão de despesas compartilhadas
            \item Rastreamento de saldos e dívidas
            \item Suporte a múltiplas moedas (100+)
            \item Digitalização de recibos (versão PRO)
            \item Importação de transações
            \item Divisão por porcentagem ou cotas
        \end{itemize}
    \item \textbf{UI/UX}: Interface limpa e funcional, disponível em 7+ idiomas
    \item \textbf{Preços}: Modelo freemium com versão PRO paga
    \item \textbf{Canais de Venda}: App stores, web, marketing digital
    \item \textbf{Reputação}: Forte presença internacional, recomendado pelo NY Times e Financial Times
    \item \textbf{Pontos Fortes}: Experiência no mercado, base de usuários estabelecida
    \item \textbf{Pontos Fracos}: Foco em divisão manual, sem integração com estabelecimentos
\end{itemize}

\subsubsection{Concorrente 2 - SplitBill}
\begin{itemize}
    \item \textbf{Funcionalidades}: Divisão básica de contas
    \item \textbf{UI/UX}: Interface simples e direta
    \item \textbf{Preços}: Aplicativo gratuito com anúncios
    \item \textbf{Canais de Venda}: App stores
    \item \textbf{Reputação}: Presença moderada no mercado
\end{itemize}

\subsubsection{Concorrente 3 - PicPay}
\begin{itemize}
    \item \textbf{Funcionalidades}: Pagamentos móveis, divisão de contas, carteira digital
    \item \textbf{UI/UX}: Interface moderna e intuitiva
    \item \textbf{Preços}: Taxas variáveis por transação
    \item \textbf{Canais de Venda}: App stores, marketing digital
    \item \textbf{Reputação}: Forte presença no mercado brasileiro
\end{itemize}

\subsection{Posicionamento Estratégico}
O Split se diferencia por:
\begin{itemize}
    \item Foco específico no mercado brasileiro
    \item Integração direta com sistemas de estabelecimentos
    \item Processamento automatizado de comandas via IA
    \item Interface simplificada e intuitiva
    \item Solução específica para o problema de divisão de contas em tempo real
    \item Foco na preservação de relacionamentos e redução de conflitos
\end{itemize}

\subsection{Vantagens Competitivas}
\begin{itemize}
    \item Reconhecimento automático de itens da comanda
    \item Divisão em tempo real durante o consumo
    \item Foco na experiência social e preservação de relacionamentos
    \item Solução adaptada à realidade e cultura brasileira
\end{itemize}

\subsection{Insights Competitivos}
\begin{itemize}
    \item Necessidade de desenvolvimento de parcerias estratégicas
    \item Oportunidade de diferenciação através da especialização
    \item Potencial para integração com sistemas existentes
    \item Demanda por soluções específicas para o setor de alimentação
\end{itemize}

\section{Análise de Mercado}

\subsection{Problema e Solução}
O mercado brasileiro de alimentação fora do lar movimenta bilhões anualmente, com milhões de pessoas frequentando restaurantes e bares diariamente. Um estudo realizado por nossa equipe revelou que 78 por cento das pessoas já experimentaram algum tipo de desconforto durante a divisão de contas em grupo, e 45 por cento admitem que já evitaram sair com certos grupos devido a experiências negativas anteriores relacionadas a este problema.

O Split resolve esta dor de forma definitiva. Nossa solução não apenas facilita a divisão de contas, mas também:
\begin{itemize}
    \item Elimina completamente o constrangimento social
    \item Reduz o tempo gasto com cálculos em média 95 por cento
    \item Previne erros comuns de matemática
    \item Mantém um histórico transparente para todos
    \item Fortalece laços sociais ao remover uma fonte comum de conflitos
\end{itemize}

\subsection{Impacto Social}
O impacto do Split vai muito além da conveniência tecnológica. Estamos ativamente contribuindo para a preservação de amizades e o fortalecimento de relacionamentos. Em nossa fase de testes, usuários relataram:

\begin{quote}
"Depois que comecei a usar o Split, as saídas com amigos ficaram muito mais leves. Não existe mais aquele momento constrangedor do final da noite." - Maria, 28 anos
\end{quote}

\begin{quote}
"Como organizador de eventos sociais, o Split revolucionou a forma como gerencio as contas do grupo. Agora posso focar em fazer as pessoas se divertirem, sem me preocupar com a matemática no final." - Carlos, 32 anos
\end{quote}

\section{Estratégia de Marketing}

\subsection{Posicionamento}
O Split se posiciona como mais que um aplicativo de divisão de contas - somos um facilitador de momentos felizes. Nossa comunicação enfatiza a preservação de amizades e a eliminação do estresse em momentos de confraternização. Diferentemente de soluções generalistas de pagamento ou aplicativos básicos de divisão, o Split entende o contexto social e emocional envolvido nas refeições em grupo.

\subsection{Mensagem Principal}
"Split - Divida a conta, multiplique as amizades"

Nossa mensagem central comunica diretamente o benefício mais valioso do produto: a capacidade de preservar e fortalecer relacionamentos ao eliminar uma fonte comum de conflitos. Queremos que as pessoas associem o Split não apenas à praticidade, mas principalmente à harmonia social.

\end{document}